\documentclass{ctexart}
\usepackage{lipsum}
\usepackage{CJKutf8}
\usepackage{amssymb}
\usepackage{amsmath}
\usepackage{amsthm}
\title{Problem Set\\线性代数问题集一}
\author{Liu Xinyi}
\date{\today}


% ---以上为头文件--- %


\begin{document}
\maketitle
\begin{abstract}
    本问题集旨在对Introduction to Linear Algebra(1)中建立的理论提供反馈练习。换言之,使用在ILA(1)中给出的概念、推论、定理和普遍的数学证明工具,能够解决本问题集中的所有问题。

    在本问题集中,按照ILA(1)建立起的理论成果,将可以解决的问题分为以下几类:

    
\end{abstract}
\newpage

\section{Problem Set}
\begin{enumerate}
   \item 设$\mathbb{R}^+$是全体正实数构成的集合,
   
   在其中定义加法和标量乘法为$+(a,b)=a\times b,\times(k,a)=a^k$,
   $\forall a,b \in \mathbb{R}^+$,$k\in \mathbb{R}$. 证明$\mathbb{R}^+$是线性空间.\\

   \vspace*{80pt}
    
   \item 设$\mathbf{a,b}$是向量空间$V^3$中的向量,证明集合$W=\{k\mathbf{a}+l\mathbf{b}\|k,l \in \mathbb{R}$是$V^3$的子空间.\\
   
   \vspace*{80pt}
    
   \item 设$V$是一个非空的向量空间,证明:$V$不能是它的两个真子空间的并集.\\
   
   \vspace*{80pt}
    
   \item 设$W_1,W_2,\cdots,W_r$是向量空间$V$的真子空间. 证明:在线性空间$V$中存在一个向量$\mathbf{\xi}$,满足
    $$
    \mathbf{\xi}\notin \cup_{1\leq i \leq r}W_i.
    $$\\

    \newpage

    \item 如果向量组$\alpha_1,\alpha_2,\cdots,\alpha_n$两两线性无关,试问$\alpha_1,\alpha_2,\cdots,\alpha_n$是否线性无关.\\
    
    \vspace*{80pt}
    
    \item 设$\alpha_1,\alpha_2,\alpha_3$线性无关,证明$\alpha_1+\alpha_2,\alpha_2+\alpha_3,\alpha_3+\alpha_1$也线性无关.\\
    
    \vspace*{80pt}
    
    \item 设$\alpha_1,\alpha_2,\cdots,\alpha_n$线性无关,并且$\beta+\alpha_1,\beta+\alpha_2,\cdots,\beta+\alpha_n$线性相关,那么$\beta$是否属于$\mathbf{L}(\alpha_1,\alpha_2,\cdots,\alpha_n)$.\\
    
    \vspace*{80pt}
    
    \item 证明$\mathbb{R}^3$中不同四点$A,B,C,D$共面的充分必要条件是存在不全为零的实数$k_1,k_2,k_3,K_4$使
    
    $$
    k_1\vec{OA}+k_2\vec{OB}+K_3\vec{OC}+k_4\vec{OD}=\vec{0}
    $$
    
    且$k_1+k_2+k_3+k_4=0$.\\

    \newpage

    \item 已知$\beta$可以被$\alpha_1,\alpha_2,\cdots,\alpha_r$线性表示,但不能被$\alpha_1,\alpha_2,\cdots,\alpha_{r-1}$线性表示.
    证明:$\alpha_1,\alpha_2,\cdots,\alpha_r$和$\alpha_1,\alpha_2,\cdots,\alpha_{r-1},\beta$等价.\\

    \vspace*{80pt}

    \item 设$\alpha_1,\alpha_2,\cdots,\alpha_n$线性无关,但$\alpha_1,\alpha_2,\cdots,\alpha_m,\beta,\gamma$线性相关. 判断$\beta,\gamma$与$\mathbf{L}(\alpha_1,\alpha_2,\cdots,\alpha_n)$的关系.\\
    
    \vspace*{80pt}
    
    \item 设$W_1,W_2$是向量空间$V$的两个非零子空间,如果$W_1\subset W_2$,且$\mbox{dim}W_1=\mbox{dim}W_2$. 证明:$W_1=W_2$.\\
    
    \vspace*{80pt}
    
    \item 假设$v_1,v_2,v_3,v_4$张成$V$,证明$v_1-v_2,v_2-v_3,v_3-v_4,v_4$也能张成$V$.\\
    
    \vspace*{60pt}
    
    \item 设$\alpha_1=(1,1,1)$,$\alpha_2=(1,2,3)$,$\alpha_3=(1,3,t)$,试问当$t$的取值如何决定向量组$\alpha_1,\alpha_2,\alpha_3$线性相关或线性无关.\\
    
    \newpage
    
    \item 设$\mathbf{W}=\{(a_1,a_2,\cdots,a_r,0,0,\cdots,0)|a_i\in \mathbb{R}\}\subset \mathbb{R}^n$. 证明$\mbox{dim }W=r$.\\
    
    \vspace*{80pt}
    
    \item 设$W_1$和$W_2$是线性空间$V$的两个非零线性子空间,如果$W_1\subset W_2$,且$\mbox{dim }W_1=\mbox{dim }W_2$. 证明:$W_1=W_2$.\\
    
    \vspace*{80pt}
    
    \item 将向量组$(2,1,-1,3),(-1,0,1,2)$扩充为$\mathbb{R}^4$的一个基.\\
    
    \vspace*{70pt}
    
    \item 设$\mathbb{R}^n$中的向量组$\alpha_1=(1,1,\cdots,1),\alpha_2=(2,2^2,\cdots,2^n),\cdots,\alpha_n=(n,n^2,\cdots,n^n)$.证明:向量组$\alpha_1,\alpha_2,\cdots,\alpha_n$是$\mathbb{R}^n$的一个张成组.\\
    
    \vspace*{70pt}
    
    \item 已知$\alpha_1=(1,2,3),\alpha_2=(-4,5,6),\alpha_3=(7,-8,9)$是$\mathbb{R}^3$的一个基,求向量$\vec{\xi}=(5,-12,3)$在这个基下的坐标.\\
    
    \newpage

    \item 在$\mathbb{R}^3$中求出向量$\alpha$,使得$\alpha$在下列两个基有相同的坐标:
    
    $\alpha_1=(1,0,1),\alpha_2=(-1,0,0),\alpha_3=(0,1,1);$
    
    $\beta_1=(0-1,1),\beta_2=(1,-1,0),\beta_3=(1,0,1).$\\
    

    \vspace*{80pt}

    \item 如果$\alpha+k\beta+\gamma=\mathbf{0}$.证明:$\mathcal{L}(\alpha,\beta)=\mathcal{L}(\beta,\gamma)$.\\
    
    \vspace*{80pt}
    
    \item 设$\mathbf{V}=\mathbb{R}^4,\mathbf{W_1}=\mathcal{L}(\alpha_1,\alpha_2,\alpha_3),\mathbf{W_2}=\mathcal{L}(\beta_1,\beta_2)$,其中
        \begin{align*}
        &\alpha_1=(1,-1,2,3), &\alpha_2=(-1,2,-1,1),\\
        &\alpha_3=(-1,0,-3,5),&\beta_1=(-1,4,0,-1),\\
        &\beta_2=(0,9,5,-14).
        \end{align*}
    求$\mathbf{W_1}$与$\mathbf{W_2}$的和与交基和维数.\\

    \newpage

    \item 设$\mathbf{W_1},\mathbf{W_2}$是有限维向量空间$\mathbf{V}$的两个子空间,并且
    $$
    \dim (\mathbf{W_1}\cap \mathbf{W_2})=\dim (\mathbf{W_1+W_2})-1.
    $$
    证明:$\mathbf{W_1}\subset \mathbf{W_2}$或者$\mathbf{W_2}\subset\mathbf{W_1}$.\\ 

    \vspace*{120pt}

    \item 求$\alpha_1=(1,-1,2,3),\alpha_2=(-1,2,-1,1)$在$\mathbb{R}^4$中的补子空间.\\
    
    \vspace*{80pt}
    
    \item 设$\mathbf{U}$是$\mathbb{R}^5$的子空间,$\mathbf{U}=\{(x_1,x_2,x_3,x_4,x_5)\in \mathbb{R}^5:x_1=3x_2,x_3=7x_4\}$.求:
    
        \textbf{(a)} $\mathbf{U}$的一个基;
        
        \textbf{(b)} 将(a)中的基扩充为$\mathbb{R}^5$的基;\\

    \newpage 

    \item 证明:若$v_1,v_2,v_3,v_4$是$\mathbf{V}$的基,$\mathbf{W}$是$\mathbf{V}$的子空间,并且$v_1,v_2\in\mathbf{W}$,$\v_3,v_4\notin \mathbf{W}$,则$v_1,v_2$是$\mathbf{W}$的基.\\
    
    \vspace*{120pt}
    
    \item 设$v_1,v_2,\cdots,v_m\in \mathbf{V}$的一组线性无关向量组,又$w\in \mathbf{V}$. 证明:
        \begin{align*}
            \dim \mathcal{L}(v_1+2,v_2+w,\cdots,v_m+w)\geq m-1.\\
        \end{align*}


\end{enumerate}

\newpage

\section{分类表与原理}

\end{document}
