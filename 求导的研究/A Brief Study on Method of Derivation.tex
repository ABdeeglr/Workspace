\documentclass{article}
\usepackage{geometry}
\geometry{a4paper}
\usepackage{ctex}
\usepackage{amsthm}
\usepackage{amsmath}
\usepackage{mathrsfs}
\usepackage{amssymb}
\usepackage{hyperref}
\usepackage{tikz-cd}
\usepackage{ytableau}
\usepackage[all]{xy}
\setcounter{section}{-1}



\pagestyle{plain}
\marginparwidth    0pt
\oddsidemargin     45pt
\evensidemargin    45pt
\topmargin         0pt
\textheight       21cm
\textwidth         13.5cm

\newtheorem{definition}{定义}[subsection]
\newtheorem{lemma}{引理}[subsection]
\newtheorem{theorem}{定理}[subsection]
\newtheorem{example}{例}[subsection]
\newtheorem{corollary}{推论}[subsection]
\newtheorem{proposition}{命题}[subsection]
\newtheorem{problem}{问题}[subsection]
\newtheorem{exer}{习题}[subsection]
\newtheorem{conj}{猜想}[subsection]

\title{Differential Calculus\\微分学}
\author{Zimmer Abdeeglr}
\date{\today}

\begin{document}
\maketitle
\begin{abstract}
    为解决求导问题而作。具体细节,思路有待填充

    研究重点1:$\lim (1+\frac1x)^x$
\end{abstract}

\newpage
%----------------------------------%










\section{(AR).Basic Concepts}
\noindent 介绍使用的基本数学概念

\newpage






















%----------------------------------%
\section{(AR).Real Number System}
\noindent 介绍实数系的构造过程

\newpage













%----------------------------------%
\section{(AR).Convergence Sequence}
\noindent 介绍用数列极限表达的实数连续性定理


\newpage















%----------------------------------%
\section{(AR).Continus Function}

\noindent 连续函数是一个数列——我自己说的.\\

\noindent 讨论函数求极限的方法和连续函数的性质(在写出更具体的思路和研究对象后删除)\\
%\newpage



\subsection{函数的收敛与发散的定义}
\newpage

\subsection{连续函数的研究}

















\newpage













%----------------------------------%
\section{(AR).Calculus}

\noindent 为了解决导数的计算问题,我们需要两类工具,一类是具体的关于各种常见函数的求导公式,需要熟记,另一类是抽象的求导法则,这位求解复杂函数的导数提供了逐步求导的原理.\\

\noindent 首先我们可以从定义出发,来证明一些基本的求导公式:

\begin{enumerate}
    \item $c'=0$,
    \item $(x^{\alpha})'=\alpha x^{\alpha-1}$,
    \item $(e^x)'=e^x$,
    \item $(a^x)'=\ln a \cdot a^x$,
    %\item $(\log_ax)'=\frac1{x\ln a}$,
    %\item $(\ln x)'=\frac1x$,
    \item $(\sin x)'=\cos x$,
    \item $(\cos x)'=-\sin x$,\\
    %\item $(\tan x)'=\frac1{\cos^2 x}$,
    %\item $(\cot x)'=-\frac1{\sin^2 x}$,
    %\item $(\arcsin x)'=\frac1{\sqrt{1-x^2}}$,
    %\item $(\arccos x)'=-\frac1{\sqrt{1-x^2}}$,
    %\item $(\arctan x)'=\frac1{1+x^2}$,
    %\item \textbf{(arcot x)'}$=\frac1{1+x^2}$,
\end{enumerate}

\noindent 这些基本的求导公式是进一步的公式推导提供了基础,接下来要介绍求导的四则运算.\\

\begin{theorem}
    设函数$f$和$g$在点$x$上可导,那么$f\pm g$,$fg$也在$x$上可导.如果$g(x)\neq 0$,则$f/g$也在$x$上可导.

    并且
    \begin{enumerate}
        \item $[f(x)\pm g(x)]'=f'(x)\pm g'(x)$;
        \item $[f(x)\cdot g(x)]'=f(x)g(x)+f(x)g'(x)$;
        \item $(\frac{f(x)}{g(x)})'=\frac{f(x)g(x)-f(x)g'(x)}{g^2(x)}$.
    \end{enumerate}
\end{theorem}

\begin{proof}
    \noindent 最直观的证明方式可以参考3Blue1Brown的视频.
\end{proof}

\newpage

\noindent 利用求导的四则运算,我们可以得到正切函数和余切函数的求导公式:

\begin{align*}
    (\tan x)'&=(\frac{\sin x}{\cos x})'\\
    &=\frac{\cos^2 x+\sin^2 x}{\cos^2 x}\\
    &=\frac{1}{\cos^2 x}\\
    \\
    (\cot x)'&=(\frac{\cos x}{\sin x})'\\
    &=\frac{-\sin^ x - \cos^2 x}{\sin^2 x}\\
    &=-\frac{1}{\sin^2 x}\\
\end{align*}

\noindent 函数除了能够四则运算之外,还有复合运算,因此也有求导的符合运算法则.这一法则被称为\textbf{链式法则}.

\begin{theorem}[复合函数的求导]
    设函数$g$在$t_0$可导,函数$f$在$x=g(t_0)$可导,则$f(g(x))$在$t_0$可导,并且
    \begin{align*}
        (f(g(t_0)))'=f'(g(t_0))g'(t_0).\\
    \end{align*}
\end{theorem}

\noindent 第三个求导法则是反函数求导法则.

\begin{theorem}[反函数的求导]
    设$y=f(x)$在$x_0$的某个小邻域上连续且严格单调.如果它在$x_0$可导,并且$f'(x_0)\neq 0$,则反函数$x=f^{-1}(y)$在$y_0=f(x_0)$可导,并且
    \begin{align*}
        (f^{-1})'(y_0)=\frac1{f'(x_0)}.\\
    \end{align*}
\end{theorem}





\newpage



\section{(AR).The Crown of Differential Calculus}
\noindent 介绍微分中值定理和泰勒展开式



\newpage



\section{(ER).The Study on $e$}



\newpage


\section{(ER).不动点研究}

\newpage


\section{(ER).周期3混沌}

\newpage






\end{document}