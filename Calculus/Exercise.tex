\documentclass{article}
\usepackage{geometry}
\geometry{a4paper}
\usepackage{ctex}
\usepackage{amsthm}
\usepackage{amsmath}
\usepackage{mathrsfs}
\usepackage{amssymb}
\usepackage{hyperref}
\usepackage{tikz-cd}
\usepackage{ytableau}
\usepackage[all]{xy}
\setcounter{section}{0}


\pagestyle{plain}
\marginparwidth    0pt
\oddsidemargin     45pt
\evensidemargin    45pt
\topmargin         0pt
\textheight       21cm
\textwidth         13.5cm

\newtheorem{exer}{习题}[section]


\title{求导练习}
\author{Liu Xinyi}
\date{\today}



\begin{document}
    
\maketitle

\section{基础练习}

\begin{exer}
    设函数$f$在$x=0$处可导,且$f(0)=0$.求极限
    $$
    \lim_{x\to 0}\frac{f(x)}{x}.\\
    $$
\end{exer}
\vspace*{80pt}


\begin{exer}
    设函数$f$在$x=0$处可导,且$a_n\to 0^-,b_n \to 0^+(n\to +\infty)$.证明:
    $$
    \lim_{n\to \infty}\frac{f(b_n)-f(a_n)}{b_n-a_n}=f'(0).\\
    $$
\end{exer}
\vspace*{80pt}

\begin{exer}
    设$f$是偶函数且在$x=0$可导.证明:$f'(0)=0$.\\
\end{exer}
\vspace*{80pt}

\begin{exer}
    设函数$f$在$x_0$可导,证明:
    $$
    \lim_{h\to 0}\frac{f(x_0+h)-f(x_0-h)}{2h}=f'(x_0).\\
    $$
\end{exer}
\vspace*{80pt}

\begin{exer}
    设函数$f$在$x=a$可导,且$f(a)\neq 0$.求数列极限
    $$
    \lim_{n\to\infty}(\frac{f(a+\frac1n)}{f(a)})^n.\\
    $$
\end{exer}
\vspace*{80pt}

\begin{exer}
    设$f(x)=x(x-1)^2(x-2)^3$.求$f'(0),f'(1),f'(2)$.\\
\end{exer}
\vspace*{80pt}

\begin{exer}
    尝试构造一个函数f,它在$(-\infty,+\infty)$上处处不可导,但
    $$
    \lim_{n\to \infty}n(f(x+\frac1n)-f(x))
    $$
    处处存在.\\
\end{exer}
\vspace*{80pt}

\end{document}