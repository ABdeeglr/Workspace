\documentclass{article}
\usepackage{geometry}
\geometry{a4paper}
\usepackage{ctex}
\usepackage{amsthm}
\usepackage{amsmath}
\usepackage{mathrsfs}
\usepackage{amssymb}
\usepackage{hyperref}
\usepackage{tikz-cd}
\usepackage{ytableau}
\usepackage[all]{xy}
\setcounter{section}{0}


\pagestyle{plain}
\marginparwidth    0pt
\oddsidemargin     45pt
\evensidemargin    45pt
\topmargin         0pt
\textheight       21cm
\textwidth         13.5cm

\newtheorem{definition}{定义}[subsection]
\newtheorem{lemma}{引理}[subsection]
\newtheorem{theorem}{定理}[subsection]
\newtheorem{example}{例}[subsection]
\newtheorem{corollary}{推论}[subsection]
\newtheorem{proposition}{命题}[subsection]
\newtheorem{problem}{问题}[subsection]
\newtheorem{exer}{习题}[subsection]
\newtheorem{conj}{猜想}[subsection]

\title{求导练习}
\author{Liu Xinyi}
\date{\today}



\begin{document}
    
\maketitle
\begin{abstract}
    为微积分求导练习而作
\end{abstract}
\newpage

\section{微分运算}

\subsection{导数的定义}
\begin{definition}
    设函数$f$在点$x_0$的近旁有定义,如果极限
    $$
    \lim_{h\to 0}\frac{f(x_0+h)-f(x_0)}{h}
    $$
    存在且有限,则称这个极限为函数$f$在点$x_0$的\textbf{导数},记作$f'(x_0)$,并称函数$f$在$x_0$\textbf{可导}.
\end{definition}

\noindent 类似于左连续和右连续,导数也可以定义左导数和右导数,分别记作$f'_-(x_0)$和$f'_+(x_0)$.\\

\noindent 对于一个定义在区间$(a,b)$上的函数$f$,如果对所有$x\in (a,b)$,函数$f$在点$x_0$可导,则称函数$f$在$(a,b)$上可导,另外如果函数$f$在端点$a$有右导数,或者在端点$b$处有左导数,则称函数$f$在$[a,b)$上可导或者在$(a,b]$上可导.\\


\begin{theorem}
    若函数$f$在$x_0$处可导,则$f$在$x_0$处连续.
\end{theorem}
\noindent 这一定理通常会被编成口诀“可导的一定连续,连续的不一定可导”.\\

\newpage

\begin{exer}
    设函数$f$在$x=0$处可导,且$f(0)=0$.求极限
    $$
    \lim_{x\to 0}\frac{f(x)}{x}.\\
    $$
\end{exer}
\vspace*{100pt}


\begin{exer}
    设函数$f$在$x=0$处可导,且$a_n\to 0^-,b_n \to 0^+(n\to +\infty)$.证明:
    $$
    \lim_{n\to \infty}\frac{f(b_n)-f(a_n)}{b_n-a_n}=f'(0).\\
    $$
\end{exer}
\vspace*{100pt}

\begin{exer}
    设$f$是偶函数且在$x=0$可导.证明:$f'(0)=0$.\\
\end{exer}
\vspace*{100pt}

\begin{exer}
    设函数$f$在$x_0$可导,证明:
    $$
    \lim_{h\to 0}\frac{f(x_0+h)-f(x_0-h)}{2h}=f'(x_0).\\
    $$
\end{exer}
\newpage


\begin{exer}
    设函数$f$在$x=a$可导,且$f(a)\neq 0$.求数列极限
    $$
    \lim_{n\to\infty}(\frac{f(a+\frac1n)}{f(a)})^n.\\
    $$
\end{exer}
\vspace*{100pt}



\begin{exer}
    设$f(x)=x(x-1)^2(x-2)^3$.求$f'(0),f'(1),f'(2)$.\\
\end{exer}
\vspace*{120pt}


\begin{exer}
    尝试构造一个函数f,它在$(-\infty,+\infty)$上处处不可导,但
    $$
    \lim_{n\to \infty}n(f(x+\frac1n)-f(x))
    $$
    处处存在.\\
\end{exer}
\vspace*{80pt}

\newpage


\noindent 在提出定义之后,我们就能对具体的问题进行计算.解决关于求导的计算问题,需要两种技术:公式和求导法则.

\subsection{求导公式}

\begin{enumerate}
    \item $c'=0$,
    \item $(x^{\alpha})'=\alpha x^{\alpha-1}$,
    \item $(e^x)'=e^x$,
    \item $(a^x)'=\ln a \cdot a^x$,
    \item $(\log_ax)'=\frac1{x\ln a}$,
    \item $(\ln x)'=\frac1x$,
    \item $(\sin x)'=\cos x$,
    \item $(\cos x)'=-\sin x$,
    \item $(\tan x)'=\frac1{\cos^2 x}$,
    \item $(\cot x)'=-\frac1{\sin^2 x}$,
    \item $(\arcsin x)'=\frac1{\sqrt{1-x^2}}$,
    \item $(\arccos x)'=-\frac1{\sqrt{1-x^2}}$,
    \item $(\arctan x)'=\frac1{1+x^2}$,
    \item \textbf{(arcot x)'}$=\frac1{1+x^2}$,

\end{enumerate}

\newpage

\subsection{求导法则}
\begin{enumerate}
    \item 四则运算法则
    \item 链式法则
    \item 反函数求导法则
    \item 洛必达法则
\end{enumerate}

\end{document}